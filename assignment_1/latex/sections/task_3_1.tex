\documentclass[../main.tex]{subfiles}

\graphicspath{{\subfix{../images/}}}

\begin{document}

\section*{Task 3.1}

Create a module (\textbf{ModuleSingle}) with a single thread and a method. The thread should notify the method each 2 ms by use of an event and static sensitivity. The method should increment a counter of the type \textbf{sc\_uint<4>} and print the value and current simulation time. Limit the simulation time to 200 ms. Describe what happens when the counter wraps around?

\subsection*{Solution}

We started by defining a simple class for \textbf{ModuleSingle}, inhereting from sc\_module.

\begin{myminted}{/inc/ModuleSingle.hpp}
class ModuleSingle : public sc_module {
public:
    ModuleSingle(sc_module_name name);
    ~ModuleSingle();

private:
    std::string moduleName;
    sc_uint<4> counter;
    sc_event event;

    void moduleSingleThread();
    void incrementCounter();
    void trigger();
};
\end{myminted}

In the constructor we register the main thread \textbf{moduleSingleThread} and the method \textbf{trigger}. The method is made sensitive to "event". Every 2 ms the thread wakes from "wait", and notifies the method.

\begin{myminted}{/src/ModuleSingle.cpp (snippet)}
ModuleSingle::ModuleSingle(sc_module_name name) 
    : sc_module(name), moduleName(name), counter(0b0000)
{
    SC_THREAD(moduleSingleThread);
    SC_METHOD(trigger);
    sensitive << event;
}

void ModuleSingle::moduleSingleThread() {
    while (true) {
        wait(2, SC_MS); 
        incrementCounter();
        event.notify();
    }
}
\end{myminted}

\newpage

The \textbf{trigger} method simple prints the time and the counter when called. A snippet of the output is shown in.


\end{document}