\documentclass[../main.tex]{subfiles}

\graphicspath{{\subfix{../images/}}}

\begin{document}

\section{Task 3.1}

Create a module (\textbf{ModuleSingle}) with a single thread and a method. The thread should notify the method each 2 ms by use of an event and static sensitivity. The method should increment a counter of the type \textbf{sc\_uint<4>} and print the value and current simulation time. Limit the simulation time to 200 ms. Describe what happens when the counter wraps around?

\subsection*{Solution}

We started by defining a simple class for \textbf{ModuleSingle}, inhereting from \texttt{sc\_module}.  The model is designed at quite a high level of abstraction. We don't simulate any pin behaviour, nor any transaction of data. The module is however approximately timed, and can therefore be considered a \textit{Timed Functional Model}.

\begin{myminted}{/inc/ModuleSingle.hpp}
class ModuleSingle : public sc_module {
public:
    ModuleSingle(sc_module_name name);
    ~ModuleSingle();

private:
    std::string moduleName;
    sc_uint<4> counter;
    sc_event event;

    void moduleSingleThread();
    void incrementCounter();
    void trigger();
};
\end{myminted}

The \textbf{moduleSingle} constructor simply registers \texttt{moduleSingleThread} and \texttt{trigger}. The method is made sensitive to \texttt{event}. Every 2 ms the thread wakes from "wait", increments the counter, and notifies the method by \texttt{event}.

\begin{myminted}{/src/ModuleSingle.cpp - moduleSingleThread()}
void ModuleSingle::moduleSingleThread() {
    while (true) {
        wait(2, SC_MS); 
        incrementCounter();
        event.notify();
    }
}
\end{myminted}

\newpage

Finally the method \texttt{trigger} simply prints the time and and current counter value.

\begin{myminted}{/src/ModuleSingle.cpp - trigger()}
void ModuleSingle::trigger() {
    std::cout << "Module " << moduleName << " - Counter: " << counter
                << " at time " << sc_time_stamp() << "\n";
}
\end{myminted}

When the counter reaches 0b1111, it "loops" back to 0b0000 due to overflow. A part of the output from ModuleSingle is shown below.

\begin{mintedterminal}{a1\_31.exe (first 16 lines)}
Module ModuleSingle_i - Counter: 0 at time 0 s
Module ModuleSingle_i - Counter: 1 at time 2 ms
Module ModuleSingle_i - Counter: 2 at time 4 ms
Module ModuleSingle_i - Counter: 3 at time 6 ms
Module ModuleSingle_i - Counter: 4 at time 8 ms
Module ModuleSingle_i - Counter: 5 at time 10 ms
Module ModuleSingle_i - Counter: 6 at time 12 ms
Module ModuleSingle_i - Counter: 7 at time 14 ms
Module ModuleSingle_i - Counter: 8 at time 16 ms
Module ModuleSingle_i - Counter: 9 at time 18 ms
Module ModuleSingle_i - Counter: 10 at time 20 ms
Module ModuleSingle_i - Counter: 11 at time 22 ms
Module ModuleSingle_i - Counter: 12 at time 24 ms
Module ModuleSingle_i - Counter: 13 at time 26 ms
Module ModuleSingle_i - Counter: 14 at time 28 ms
Module ModuleSingle_i - Counter: 15 at time 30 ms
\end{mintedterminal}

\end{document}