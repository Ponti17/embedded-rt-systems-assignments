\documentclass[../main.tex]{subfiles}

\graphicspath{{\subfix{../images/}}}

\begin{document}

\section{Task 7}

In this task, we will use HLS to convert a SystemC hardware description to an RTL implementation.

\begin{itemize}
    \item Write the ADVIOS IP core and a testbench in SystemC.
    \item Document and verify the result of the simulation and synthesis using Vivado HLS.
    \item Connect the ctrl port to the AXI4Lite interface by using pragma as described in UG902, p. 161.
    \item Create a Vivado project and add the ADVIOS IP core and connect it to the LEDS and SWITCHES on the ZYBO board.
    \item Write a program with the Xilinx SDDK that verifies the functionality of the IP core and document the results.
\end{itemize}

\subsection*{Solution}

% Something something i don't know
\lipsum[1]

\newpage

We had the \textit{classic} of issue of Vivado not wanting to generate the necessary files for the IP core in the BSP. To read/write to the IP core register, we chose to simply use \texttt{Xil\_Out8}. We found the driver file of the IP core, and looked up the register offset to be \texttt{0x14}. A test program was written where different values could be written to the \texttt{ctrl} register. In addition we also added to capability to read the value of the register. The main event loop of the program is shown below.

\begin{myminted}{main.c - Main Event Loop}
while (1)
{
    xil_printf("\r\n\nCMD:> ");
    input = inbyte();
    xil_printf("%c", input);
    switch (input)
    {
        case '0':
            xil_printf("\r\nReading CTRL: ");
            ctrl_read = Xil_In8(CTRL_ADDR) & 0xF;
            xil_printf("%u", ctrl_read);
            break;
        case '1':
            xil_printf("\r\nWriting 0x0.");
            Xil_Out8(CTRL_ADDR, 0x0);
            break;
        case '2':
            xil_printf("\r\nWriting 0xA.");
            Xil_Out8(CTRL_ADDR, 0xA);
            break;
        case '3':
            xil_printf("\r\nWriting 0xF.");
            Xil_Out8(CTRL_ADDR, 0xF);
            break;
        case '4':
            xil_printf("\r\nGenerating random value.");
            uint8_t rand = rand_uint8() & 0xF;
            xil_printf("\r\nWriting %u to CTRL.", rand);
            Xil_Out8(CTRL_ADDR, rand);
            break;
        default:
            xil_printf("\r\nUnrecognized input. \"%c\"", input);
            break;
    }
}
\end{myminted}

\end{document}